\documentclass[14pt]{extreport}
\usepackage{gost}



\begin{document}
\includepdf[pages={1}]{titulOzPrac.pdf}


\tableofcontents

\intro

Ознакомительная практика является неотъемлемой частью учебного процесса. Студентам необходима подготовка к осознанному и углубленному изучению общепрофессиональных и специальных дисциплин и получение навыков самостоятельной практической работы по сбору фактического материала, составлению базы данных для различных исследований, анализа собранной информации.




\chapter{Множества. Действительные числа}

\section{Основные понятия}

Под множеством понимают совокупность (собрание, класс, семейство...) некоторых объектов, объединенных по какому-либо признаку.
Множество обозначается заглавными буквами, например $M$, $X$. Прописными латинскими буквами обозначаются элементы множеств, например $a$, $x$.
\begin{itemize}
\item $a \in M$ - элемент $a$ принадлежит множеству $M$;
\item $a \notin M$ - элемент $a$ не принадлежит множеству $M$;
\item $\exists$ - <<существует>>;
\item $\exists x \in M$ - существует $x$ из $M$;
\item $\forall$ - <<для любого>>, <<для всякого>>;
\item $a \wedge b$ - конъюнкция (<<и>>) - логическое умножение;
\item $a \vee b$  - дизъюнкция (<<или>>) - логическое сложение;
\item $\neg$ - <<не>> - логическое отрицание, например $\neg \forall x \in M$ - не для всех $x$ из $M$
\item $:$ - <<имеет место>>, <<такое что>>.
\end{itemize}



\section{Числовые множества. Множества действительных чисел}

Множества, элементами которых являются числа, называются \emph{числовыми}. Примерами числовых множеств являются:
\begin{itemize}
\item $N = \{1; 2; 3; ...; n; ...\}$ - множество натуральных чисел;
\item $Z_0 = \{0; 1; 2; ...; n; ...\}$ - множество целых неотрицательных чисел;
\item $Z = \{0; \pm 1; \pm 2; ...; \pm n; ...\}$ - множество целых чисел;
\item $Q = \{\frac mn : m \in Z, n \in N\}$ - множество рациональных чисел;
\item $R$ - множество действительных чисел.
\end{itemize} 

Действительные числа, не являющиеся рациональными, называются \emph{иррациональными}.



\section{Числовые промежутки. Окрестность точки}
\begin{example}
Пусть $x_0$ - любое дейтвительное число (точка на числовой прямой). \emph{Окрестностью} точки $x_0$ называется любой интервал $(a; b)$, содержащий точку $x_0$. В частности, интервал $(x_0 - \varepsilon, x_0 + \varepsilon)$, где $\varepsilon > 0$, называется $\varepsilon$-окрестностью точки $x_0$. Число $x_0$ называется \emph{центром}, а число $\varepsilon$ - \emph{радиусом}.

\begin{figure}[H]
\centerline{\includegraphics[width=1.0\linewidth]{pic1}}
\caption{Проверка точного решения}
\label{fig11}
\end{figure}

Если $x \in (x_0 - \varepsilon; x_0 + \varepsilon)$, то выполняется неравенство $x_0 - \varepsilon < x < x_0 + \varepsilon$, или, что то же, $|x - x_0| < \varepsilon$. Выполнение последнего неравенства означает попадание точки $x$ в $\varepsilon$-окрестность точки $x_0$ (см. рис. 1.1).
\end{example}





\chapter{Функция}


\section{Числовые функции. График функции}

\begin{example}
$f(x)=3x^2$
\begin{figure}[H]
\centerline{\includegraphics[width=1.0\linewidth]{gra-001}}
\caption{График функции - параболла}
\label{fig12}
\end{figure}
\end{example}

\begin{example}
$f(x)=|x|$
\begin{figure}[H]
\centerline{\includegraphics[width=1.0\linewidth]{gra-002}}
\caption{График функции - $f(x)=|x|$}
\label{fig13}
\end{figure}
\end{example}


\section{Обратная функция}
\begin{example}
Найти функцию обратную для $y=3x+2$

Выразим $x$ через $y$, получаем: 
\begin{equation}
x = \frac{1}{3}y-\frac{2}{3} 
\end{equation}

Это и есть обратная функция, переставив буквы $x$ и $y$, будем писать: $y=\frac{1}{3}x-\frac{2}{3}$.

Таким образом, $y=3x+2$ и $y=\frac{1}{3}x-\frac{2}{3}$ - взаимно обратные функции.
\begin{figure}[H]
\centerline{\includegraphics[width=1.0\linewidth]{gra-003}}
\caption{}
\label{fig14}
\end{figure}
\end{example}


\chapter{Последовательности}


\section{Предел числовой последовательности}
\begin{example}

Докажем, что:

$$\lim\limits_{n\to \infty}\frac{-1^n}{n}=0$$

Запишем определение предела: 
$$\forall \varepsilon<0 \exists n_0 = n(\varepsilon):\forall n \geqslant n_0 $$
$$\arrowvert\frac{-1^n}{n}-0\arrowvert<\varepsilon\Longrightarrow\frac{1}{n}<\varepsilon\Longrightarrow n(\varepsilon)>\frac{1}{\varepsilon}$$
$$n_0=[\frac{1}{\varepsilon}]+1$$
т.е. все члены последовательности начиная с такого $n_0$, лежат в интервале $(-\varepsilon; +\varepsilon)$.
\end{example}

\section{Предел монотонной ограниченной последовательности. Число $e$. Натуральные логарифмы}
\begin{theorem}
Последовательность с общим членом $e_n=(1+\frac{1}{n})^n$ имеет конечный предел при $n\rightarrow\infty$.
\end{theorem}
\begin{proof}

Покажем сначала, что $\{e_n\}$ представляет собой монотонно возрастающую последовательность. Согласно биному Ньютона, полагая $a=1, b=\frac{1}{n}$, получим

$$e_n=(1+\frac{1}{n})^n=1+n\cdot\frac{1}{n}+\frac{n(n-1)}{2!}\cdot\frac{1}{n^2}+\frac{n(n-1)(n-2)}{3!}\cdot\frac{1}{n^3}+\dots+\frac{1}{n^n}=$$ $$=2+\frac{1}{2!}(1-\frac{1}{n})+\frac{1}{3!}(1-\frac{1}{n})(1-\frac{2}{n})+\dots+\frac{1}{n!}(1-\frac{1}{n!}\dots(1-\frac{n-1}{n}).$$

Аналогично,
$$e_n+1=(1+\frac{1}{n+1})^N+1=2+\frac{1}{2!}(1-\frac{1}{n+1})+\frac{1}{3!}(1-\frac{1}{n+1})(1-\frac{2}{n+1})+\dots$$

Далее докажем, что последовательность $\{e_n\}$ является ограниченной. Действительно, первый член любой монотонно возрастающей последовательности является ее наибольшей нижней границей и, таким образом, $e_n\geqslant 2$ для всех натуральных значений $n$. Перейдем к доказательству существования верхней границы. Очевидно, что
$$e_n=2+\frac{1}{2!}(1-\frac{1}{n})+\frac{1}{3!}(1-\frac{1}{n})(1-\frac{2}{n})+\dots+\frac{1}{n!}(1-\frac{1}{n})\dots(1-\frac{n-1}{n})<$$ $$<2+\frac{1}{2!}+\frac{1}{3!}+\dots+\frac{1}{n!}.$$

Кроме того, $\frac{1}{k!}<\frac{1}{2^k}$ для всех $k>3$. Тогда
$$\frac{1}{4!}+\frac{1}{5!}+\dots+\frac{1}{n!}<\frac{1}{2^4}+\frac{1}{2^5}+\dots+\frac{1}{2^n}.$$

Правая часть этого неравенства представляет собой сумму членов убывающей геометрической прогрессии. В качестве верхней границы этой суммы выступает любое число $U\geqslant\frac{1}{8}.$ Таким образом, последовательность с общим членом
$$e_n=2+\frac{1}{2!}+\frac{1}{3!}+\dots+\frac{1}{n!}<2+\frac{1}{2}+\frac{1}{6}+\frac{1}{8}<3$$
представляет собой ограниченную монотонно возрастающую последовательностьи, следовательно, имеет конечный предел - согласно теореме о мнотонных последовательностях.

\end{proof}



\chapter{Предел функции}

\section{Предел функции в точке}
\begin{example}

На <<языке $\varepsilon - \delta,$>> или по Коши.

Используя $\varepsilon$ - $\delta$ - определение предела, показать что $\lim\limits_{x \to 3}(3x-2)=7.$

\emph{Решение.}
 Пусть $\varepsilon>0$ является произвольным положительным числом. Выберем $\delta=\frac{\varepsilon}{3}.$ Очевидно, что если 

$0<|x-3|<\delta,$

 то

$|f(x)-L|=|(3x-2)-7|=|3x-9|=3|x-3|<3\delta=3\cdot\frac{\varepsilon}{3}=\varepsilon.$

Данный предел доказан в соответствии с определением Коши.

\end{example}

\begin{example}
На <<языке последовательностей>>, или по Гейне.

Доказать, что $f(x)=\sin\frac{1}{x}$ не имеет предела в точке $0$.
$$\forall\{x_n'\}\rightarrow 0\exists\{x_n\} \rightarrow0$$
$$\{f(x_n')\}\rightarrow A_1\{f(x_n)\}\rightarrow A_2$$
$$x_n':\sin\frac{1}{x_n'}=0\Leftrightarrow\frac{1}{x_n'}=\pi n\Longrightarrow x_n'=\frac{1}{\pi n} \xrightarrow[n\neq 0]{n\rightarrow \infty}0$$
$$x_n'=\frac{1}{\pi n}\rightarrow0:f(x_n')=0\xrightarrow{n\neq0}0$$
$$x_n:\sin\frac{1}{x_n}=1\Leftrightarrow\frac{1}{x_n}=\frac{\pi}{2}+2\pi n \Longrightarrow x_n=\frac{1}{\frac{\pi}{2}+2\pi n}\xrightarrow[n\neq0]{n\rightarrow\infty}0$$
$$x_n=\frac{1}{\frac{\pi}{2}+2\pi n}\rightarrow0:f(x_n)=1\rightarrow1$$

Последовательность по Гейне не имеет предела.
\end{example}

\section{Односторонние пределы}

\section{Предел функции при $x\to\infty$}

\section{Бесконечно большая функция}



\chapter{Бесконечно малые функции}

\section{Определения и основные теоремы}

\section{Связь между функцией, ее пределом и бесконечно малой функцией}

\section{Основные теоремы о пределах}

\section{Признаки существования пределов}

\section{Первый замечательный предел}

\section{Второй замечательный предел}



\chapter{Эквивалентные бесконечно малые функции}

\section{Сравнение бесконечно малых функций}

\section{Эквивалентные беконечно малые и основные теоремы о них}

\section{Применение эквивиалентных бесконечно малых функций}



\chapter{Непрерывность функций}

\section{Непрерывность функции в точке}

\section{Непрерывность функции в интервале и на отрезке}

\section{Точки разрыва функции и их классификация}

\section{Основные теоремы о непрерывных функциях. Непрерывность элементарных функций}

\section{Свойства функций непрерывных на отрезке}



\chapter{Производная функции}


























\end{document}